\chapter{Completion and deployment}

\subsection{Deployment to production}
Once the application had been through several development iterations and was nearing completion, it was presented to the researchers for signing off, and discussions about deployment and maintenance. A couple of changes were requested, first a splash page that provided a place for information about the project to be listed, secondly, they wanted the website to be password protected. 

These changes were quickly implemented and the project was deployed to a Virtual Machine (VM) on the universities network. To ensure the stability of the service, it was put behind a Varnish\cite{varnish} cache.

Doing this enabled caching on all of the responses from the webserver to client requests, meaning that if a request has already had a response processed for it, then a cached version will be served from Varnish rather than adding load to the NodeJS application. This speeds up the site as data can be immediately returned, once it has been processed already.

A systemd\cite{systemd} service file was also written and installed on the server for the NodeJS process, this enables the application to be controlled from systemd. This is done so that the application can be started at boot time, as well as monitored, restarted and stopped from the systemd interface. 

One difficulty faced during this deployment was the versioning on MongoDB, provided by the OS on the VM. The application had been developed on Arch Linux which provides a recent version (3.4) of MongoDB in it's package manager, however on the VM which was running Ubuntu, the package was for version 2.6, that doesn't support the new wiredTiger\cite{tiger} storage engine. 

Initially it was spotted that the MongoDB version was different, and corrected it by installing version 3.4. However when data was entered into the database, it was taking up around three times as much room, filling up the VM's disk space. 

To correct this you have to manually update the storage engine and create a new database directory in the MongoDB configuration. Once this was done the data was imported correctly, taking up the expected amount of storage. 

% This section should discuss issues you encountered as you tried to implement your experiments. What were the results of running the experiments? What conclusions can you draw from these results? 

% During the work, you might have found that elements of your experiments were unnecessary or overly complex; perhaps third party libraries were available that simplified some of the functions that you intended to implement. If things were easier in some areas, then how did you adapt your project to take account of your findings?

% It is more likely that things were more complex than you first thought. In particular, were there any problems or difficulties that you found during implementation that you had to address? Did such problems simply delay you or were they more significant? 

% If you had multiple experiments to run, it may be sensible to discuss each experiment in separate sections. 
