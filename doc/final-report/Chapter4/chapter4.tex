\chapter{Results and Conclusions}
I can't really do this section.

Maybe talk about deployment to the Uni servers, and how the researchers will use it? 

two options:

\subsection{Deployment and future work}
Due to the commercially sensitive nature of the data being used in the project, this service cannot be hosted on a publicly accessible server. Because of this it is planned to be deployed to a Virtual Machine running on the Universities servers once the project is complete. 

After demonstrating the application to the researchers, it was made clear that they would prefer it if the site was put behind a password to ensure that they had control over the data within the university. To do this there could either be a simple user system developed, or as a simple solution a HTTP basic auth login check added to the website. The later solution is very easy to implement, one extra package and one line of code, when compared to the users system, that would require front end changes, however it is a lot less flexible. 

After the demonstration they also highlighted how they would like a splash page containing information about the project, the institution and the researchers themselves. This would not be difficult to implement, it would just mean adding another route to the application, however it depends on them providing the information and design for how the page should be presented. This is a low priority as it is mainly for when the project is published and made available to the public, which could be a year or more away from happening. 

\subsection{Deployment to production}
Once the application had been through several development iterations and was nearing completion, it was presented to the researchers for signing off, and discussions about deployment and maintenance. A couple of changes were requested, first a splash page that provided a place for information about the project to be listed, secondly, they wanted the website to be password protected. 

These changes were quickly implemented and the project was deployed to a Virtual Machine on the universities network. To ensure the stability of the service, it was put behind a Varnish\cite{varnish} cache.

Doing this enabled caching on all of the responses from the webserver to client requests, meaning that if a request has already had a response processed for it, then a cached version will be served from Varnish rather than adding load to the NodeJS application. This speeds up the site as data can be immediately returned, once it has been processed already.

A systemd\cite{systemd} service file was also written and installed on the server for the NodeJS process, this enables the application to be controlled from systemd. This is done so that the application can be started at boot time, as well as monitored, restarted and stopped from the systemd interface. 

% This section should discuss issues you encountered as you tried to implement your experiments. What were the results of running the experiments? What conclusions can you draw from these results? 

% During the work, you might have found that elements of your experiments were unnecessary or overly complex; perhaps third party libraries were available that simplified some of the functions that you intended to implement. If things were easier in some areas, then how did you adapt your project to take account of your findings?

% It is more likely that things were more complex than you first thought. In particular, were there any problems or difficulties that you found during implementation that you had to address? Did such problems simply delay you or were they more significant? 

% If you had multiple experiments to run, it may be sensible to discuss each experiment in separate sections. 
