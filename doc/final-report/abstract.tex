\thispagestyle{empty}

\begin{center}
    {\LARGE\bf Abstract}
\end{center}

Understanding data produced by next gen sequencing technologies is a difficult problem, not helped by a lack of well designed software and user experiences. 

Researchers at Aberystwyth University are looking into the possibly of modifying a species of yeast, \textit{Candida tropicalis}, to convert xylose into xylitol. 

If xylitol can be produced by \textit{C. tropicalis} it would provide a cheaper supply of the sugar, which has antibacterial properties, as well as a very low glycaemic index, making it suitable for people suffering from diabetes.

To do this they have sequenced the DNA of three species of yeast, \textit{C. tropicalis, C. shehate} and \textit{C. boidinii}, but now need an accessible and meaningful way to interact with the data produced.

Most genome database software is already over a decade old, and continue to use dated practices, and technologies. 

The aim for this project is to apply modern web development standards to this problem and produce a web site to display their data using NodeJS, with the data stored in a NoSQL database. 

The implementation of this database and website has shown that NoSQL is a viable approach to storing and viewing genomic information, especially for smaller genomes. However it does also acts as a proof of concept and provides a step stone for the building of a larger open source project that aims to accommodate larger genomes. 

Importantly it shows that modern technologies and approaches can have a meaningful benefit over the old monolithic solutions, especially in terms of producing high quality maintainable software.
